%%%%%%%%%%%%%%%%%%%%%%%%%%%%%%%%%%%%%%%%%%%%%%%%%%%%%%%%%%%%%%%%%%%%%
%
% LaTeX Template
%
% This is a LaTeX document. LaTeX is a markup language for producing 
% documents. You will complete this document, compile it into a PDF 
% document, then upload it to the submission system. 
%
% To compile into a PDF on department machines:
% > pdflatex thisfile.tex
%
% If you do not have LaTeX, your options are:
% - Personal laptops (all common OS): http://www.latex-project.org/get/ 
%   + VSCode extension: https://marketplace.visualstudio.com/items?itemName=James-Yu.latex-workshop
% - Online Tool: https://www.overleaf.com/ - most LaTeX packages are pre-installed here (e.g., \usepackage{}).
%
% If you need help with LaTeX, please come to office hours.
% Or, there is help online:
% https://en.wikibooks.org/wiki/LaTeX
%
% Good luck!
%
%%%%%%%%%%%%%%%%%%%%%%%%%%%%%%%%%%%%%%%%%%%%%%%%%%%%%%%%%%%%%%%%%%%%%

\documentclass[11pt]{article}

\usepackage[english]{babel}
\usepackage[utf8]{inputenc}
\usepackage[colorlinks = true,
            linkcolor = blue,
            urlcolor  = blue]{hyperref}
\usepackage[a4paper,margin=1.5in]{geometry}
\usepackage{stackengine,graphicx}
\usepackage{fancyhdr}
\setlength{\headheight}{15pt}
\usepackage{microtype}
\usepackage{booktabs}
\usepackage{times}
\usepackage[shortlabels]{enumitem}
\setlist[enumerate]{topsep=0pt}
% a great python code format: https://github.com/olivierverdier/python-latex-highlighting
\usepackage{pythonhighlight}
\usepackage{amssymb}
\usepackage{multicol}
% setup for todo lists:
\usepackage{enumitem}
\newlist{todolist}{itemize}{2}
\setlist[todolist]{label=$\square$}
\usepackage{pifont}
\newcommand{\cmark}{\ding{51}}%
\newcommand{\done}{\rlap{$\square$}{\raisebox{2pt}{\large\hspace{1pt}\cmark}}%
\hspace{-2.5pt}}

\frenchspacing
\setlength{\parindent}{0cm} % Default is 15pt.
\setlength{\parskip}{0.3cm plus1mm minus1mm}

\pagestyle{fancy}
\fancyhf{}
\lhead{Project 2 Questions}
\rhead{CSCI 1290}
\rfoot{\thepage}

\date{}

\title{\vspace{-1cm}Project 2 Questions}


\begin{document}
\maketitle
\vspace{-3cm}
\thispagestyle{fancy}

\section*{Template Instructions}

This document is a template with specific answer regions and a fixed number of pages.
Given limited TA time, the template helps the course staff to grade efficiently and still focus on the content of your submissions. Please help us in this task:
 
\begin{itemize}
  \item Make this document anonymous.
  \item Compile the document without any answers and see the page layout.
  \item Add your answers.
  \item Maintain the page layout---your answers should not break a new page and change which question lands on which page.

  \item If you need extra pages, e.g., for images, please put them at the back of the document.
\end{itemize}

\section*{Gradescope Submission}
\begin{itemize}
  \item Compile this document to a PDF and submit it to Gradescope.
  \item Pages will be automatically assigned to the right questions on Gradescope to ease TA workload.
\end{itemize}

\section*{Instructions}
\begin{itemize}
  \item Complete the four questions.
  \item Include code, images, or equations where appropriate.
\end{itemize}


%%%%%%%%%%%%%%%%%%%%%%%%%%%%%%%%%%%

% Please leave the pagebreak
\pagebreak
\section*{Questions}

\paragraph{Q1:} 
When capturing a set of images (a \emph{bracket}) for high-dynamic range radiance map recovery, the camera's control and various photography accessories present us with different options for varying the exposure in a measured way. Describe:
\begin{enumerate}
  \item What options the camera and various photography accessories presents to us to vary exposure,
  \item What benefits or drawbacks each of those options presents,
  \item How easy each of those drawbacks might be to overcome computationally, and a brief description of what that would entail to your understanding.
\end{enumerate}

There are at least three options. Are there more than three?

%%%%%%%%%%%%%%%%%%%%%%%%%%%%%%%%%%%
\paragraph{A1:} Your answer here.




%%%%%%%%%%%%%%%%%%%%%%%%%%%%%%%%%%%
% Please leave the pagebreak
\pagebreak
\paragraph{Q2:} 
Once we've recovered a high dynamic range radiance map that represents very large exposure variations in a single image, we must store it efficiently. This requires us to use a new storage method for our images as the intensity variations no longer fit within an unsigned 8-bit integer per pixel per channel (1 Mpixel image = 3 MB). One candidate option is to store a double floating point number per pixel per channel (1 Mpixel image = 24 MB).

Can you do better? Design a pixel encoding scheme for HDR images, and explain why you think this is a good solution among possible designs.

\emph{Points to consider:} numerical precision, storage requirements, CPU/GPU register size, computational efficiency.

\emph{Points that you are not \emph{required} to consider:} compression.

\emph{Hint:} The lecture material has a few slides on this topic, and you are free to research common existing formats (like the RAW images we captured in lab 2). Use your broader CS knowledge, but there's no need to go too deep.

%%%%%%%%%%%%%%%%%%%%%%%%%%%%%%%%%%%
\paragraph{A2:} Your answer here.





%%%%%%%%%%%%%%%%%%%%%%%%%%%%%%%%%%%
\pagebreak
\paragraph{Q3a:} 
The stretch goal task 3 in Lab 3 looks at the \emph{cross bilateral filter} (sometimes called a joint bilateral filter); the lab explains how it can be used for flash/no-flash photography to reduce noise in low light.

Using your implementation of the bilateral filter, complete task 3 and attempt to combine a high-noise low-light image with a low-noise high-light (flash) image. You might have to tweak your implementation, and find appropriate $\sigma_r$ and $\sigma_s$ values. 

Note: wherever the flash image is smooth, it's OK to blur. So, shadows from flash might be a problem as they \emph{introduce} edges where none existed in the low-light image. Clever scene selection can avoid this.

\emph{This will be your 'interesting image' for this project.}

Please upload your before + after photos to this Google Drive: \href{https://tinyurl.com/csci1290fall2023}{https://tinyurl.com/csci1290fall2023}. We will share the images with the class.

%%%%%%%%%%%%%%%%%%%%%%%%%%%%%%%%%%%
\paragraph{A3a:} Your answer here.



%%%%%%%%%%%%%%%%%%%%%%%%%%%%%%%%%%%
\pagebreak
\paragraph{Q3b:}
In general, the idea of using statistics from across multiple aligned images is powerful. What other application of the cross bilateral filter can you think of? How might it be applied, and what are its effects? 

Consider the many sources of information in images: low/high dynamic range images, IR (+thermal) images, UV images, terahertz or xray images, low/high resolution images, sparsely-sampled images (LIDAR), adding other kinds of light sources to a scene (e.g., projected light), depth images, multiple cameras, video.

For inspiration, please feel free to research for techniques, and include example images here (with citation) if you use this method.


%%%%%%%%%%%%%%%%%%%%%%%%%%%%%%%%%%%
\paragraph{A3b:} Your answer here.








% %%%%%%%%%%%%%%%%%%%%%%%%%%%%%%%%%%%


% If you really need extra space, uncomment here and use extra pages after the last question.
% Please refer here in your original answer. Thanks!
%\pagebreak
%\paragraph{AX.X Continued:} Your answer continued here.



\end{document}
